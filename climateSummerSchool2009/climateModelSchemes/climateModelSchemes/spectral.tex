%%%%%%%%%%%%%%%%%%%%%%%%%%%%%%%%%%%%%%%%%%%%%%%%%%%%%%%%%%%%%%%%%%%%%%%%%%%%%%%%%
\begin{slide}{The Spectral Method}
%%%%%%%%%%%%%%%%%%%%%%%%%%%%%%%%%%%%%%%%%%%%%%%%%%%%%%%%%%%%%%%%%%%%%%%%%%%%%%%%%

\ \\
\begin{list0}

\item As high order accuracy as is possible on any particular grid\\
$\rightarrow$ best accuracy for the computational effort FOR SMOOTH solutions\\
$\rightarrow$ particularly accurate for low resolution

\item Not conservative

\item Requires a periodic domain $\therefore$ not ideal for limited area models

\item Suffers from spectral ringing
\begin{list1}
    \item oscillations either side of discontinuities due to representation of all fields as sums of waves
\end{list1}
\item Parallelisability limited due to global communications of decomposing solution into waves.\\
But ECMWF benchmarks $\rightarrow$ purchase of parallel rather than vector machine.
\end{list0}

\end{slide}

%%%%%%%%%%%%%%%%%%%%%%%%%%%%%%%%%%%%%%%%%%%%%%%%%%%%%%%%%%%%%%%%%%%%%%%%%%%%%%%%%
\begin{slide}{The Spectral Method}
%%%%%%%%%%%%%%%%%%%%%%%%%%%%%%%%%%%%%%%%%%%%%%%%%%%%%%%%%%%%%%%%%%%%%%%%%%%%%%%%%

\ \\
Express the state, $\phi$, of the atmosphere as a sum of $N$ waves of different waves lengths. 1D Example:

\[
\phi(x,t) = \sum_{k=0}^{N} a_k \sin kx + b_k \cos kx = \sum_{k=0}^N A_k e^{ikx}
\]

where $x\in[0, 2\pi]$ and $a_k$, $b_k$ or complex valued $A_k$ are the Fourier coefficients.

Fourier coefficients found using an fft -- fast Fourier transform

$N$ is the spectral resolution or spectral truncation

Remember use for Fourier filtering:\\
\begin{minipage}{0.32\linewidth}
\includegraphics[width=\linewidth]{Fourier/realSpace.pdf}
\end{minipage}
%
\begin{minipage}{0.32\linewidth}
\includegraphics[width=\linewidth]{Fourier/firstModes.pdf}
\end{minipage}
%
\begin{minipage}{0.32\linewidth}
\includegraphics[width=\linewidth]{Fourier/lastModes.pdf}
\end{minipage}

\begin{minipage}[t]{0.38\linewidth}\centering
Black line -- original data $\rightarrow$
\end{minipage}
%
\begin{minipage}[t]{0.23\linewidth}\centering
First few modes
\end{minipage}
%
\begin{minipage}[t]{0.37\linewidth}\centering
Highest frequency modes
\end{minipage}
\end{slide}

%%%%%%%%%%%%%%%%%%%%%%%%%%%%%%%%%%%%%%%%%%%%%%%%%%%%%%%%%%%%%%%%%%%%%%%%%%%%%%%%%
\begin{slide}{Equation to Solve using the Spectral Method}
%%%%%%%%%%%%%%%%%%%%%%%%%%%%%%%%%%%%%%%%%%%%%%%%%%%%%%%%%%%%%%%%%%%%%%%%%%%%%%%%%

\ \\
The fast moving gravity (or sound) waves in the atmosphere are controlled by a Helmholtz equation.

The momentum, continuity and Perfect gas equations are combined:

\setlength{\tabcolsep}{12pt}
\renewcommand{\arraystretch}{2}
\begin{tabular}{lcl}
Momentum:
&
$\frac{\partial\rho\vec{U}}{\partial t} + \nabla\dprod\rho\vec{U}\vec{U} + \nabla p = \text{other terms}$
&
Take divergence
\\
Continuity:
&
$\frac{\partial\rho}{\partial t} + \nabla\dprod\rho\vec{U} = 0$
&
Take rate of change
\\
Perfect Gas:
&
$\rho=\frac{p}{RT}$
&
Substitute into $\frac{\partial\rho}{\partial t}$ term of continuity
\end{tabular}

Momentum and continuity now both contain $\nabla\dprod\frac{\partial\rho\vec{U}}{\partial t}$ so they are combined to form the Helmholtz equation for pressure:
\[
\frac{\partial^2 \frac{p}{RT}}{\partial t^2} - \nabla^2 p = \nabla\dprod\nabla\dprod\rho\vec{U}\vec{U} + \nabla\dprod\text{other terms}
\]
LHS: Helmholtz part: controls fast waves\\
Semi-implicit method: solve Helmholtz part implicitly using values from previous time step for slower terms on RHS (explicit)
\end{slide}

%%%%%%%%%%%%%%%%%%%%%%%%%%%%%%%%%%%%%%%%%%%%%%%%%%%%%%%%%%%%%%%%%%%%%%%%%%%%%%%%%
\begin{slide}{Example: Solution of 1D Helmholtz Equation using Spectral Method}
%%%%%%%%%%%%%%%%%%%%%%%%%%%%%%%%%%%%%%%%%%%%%%%%%%%%%%%%%%%%%%%%%%%%%%%%%%%%%%%%%
\[
\frac{\partial^2 \phi}{\partial t^2} - \Phi \frac{\partial^2 \phi}{\partial x^2}
= F(x)
\]

\begin{list0}
\item Express the state vector $\phi(x,t)$ as a sum of Fourier coefficients:
\vspace{-4ex}\[
\phi(x,t) = \sum_{k=0}^N A_k(t) e^{ikx}
%
\implies \frac{\partial^2 \phi}{\partial x^2} = -\sum_{k=0}^N k^2 A_k e^{ikx}
\text{ and }
\frac{\partial^2 \phi}{\partial t^2} = \sum_{k=0}^N k^2 \frac{dA_k}{dt} e^{ikx}
\]\vspace{-4ex}
\item Substituting these into the Helmholtz equation gives O.D.E.s:
\vspace{-4ex}\[
\sum_{k=0}^N\biggl( \frac{d^2 A_k}{dt^2} + \Phi k^2 A_k\biggr) e^{ikx} = F(x)
\]
where $F(x)$ contains the interactions between different wave numbers.\\
\item For a semi-implicit numerical solution we assume that $F(x)$ is not dependent on time (explicit part)\\
$\therefore$ solve $\frac{d^2 A_k}{dt^2} + \Phi k^2 A_k = \text{fft}_k(F)$ numerically for each $k$, for example using:
\vspace{-4ex}\[
\frac{d^2 A_k}{dt^2} \approx
\frac
{\frac{A_k^n - A_k^{n-1}}{\delta t} - \frac{A_k^{n-1} - A_k^{n-2}}{\delta t}}
{\delta t}
\text{ where $n$ represents the time level}
\]
This gives solutions for each $A_k^n$
\item Inverse Fourier transforms from $A^n_k$ used to get $\phi^n(x) = \phi(x,t)$
\end{list0}
\end{slide}


