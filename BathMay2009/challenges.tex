%%%%%%%%%%%%%%%%%%%%%%%%%%%%%%%%%%%%%%%%%%%%%%%%%%%%%%%%%%%%%%%%%%%%%%%
\begin{slide}{Challenges}
%%%%%%%%%%%%%%%%%%%%%%%%%%%%%%%%%%%%%%%%%%%%%%%%%%%%%%%%%%%%%%%%%%%%%%%

\begin{list0}

\item Adaptive Meshing of:
\begin{minipage}[t]{0.5\linewidth}\begin{list2m}
    \item Deep convection
    \item Tropical cyclones
    \item Orography
    \item Fronts
\end{list2m}\end{minipage}

\item Unstructured or block-structured mesh adaptation

\item Efficiency and accuracy

\item Adaptation criteria and adaptation frequency

\item Mapping under adaptation, compromise between:
\begin{minipage}[t]{0.3\linewidth}\begin{list2m}
    \item Efficiency
    \item Balance
    \item Conservation
\end{list2m}\end{minipage}

\item Ever more parallelisable algorithms

\item Physical parametrisations that are accurate over a wide range of resolutions and do not spuriously trigger or inhibit mesh adaptivity

\item Data assimilation on a multi-resolution mesh


\end{list0}

\end{slide}