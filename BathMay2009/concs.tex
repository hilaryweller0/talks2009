\begin{slide}{Conclusions}

\begin{list0}

\item Accuracy of AtmosFOAM compares well with standard spectral model

\item Voronoi meshes with gradual refinement are particularly beneficial

\item Abrupt local mesh refinement can degrade global accuracy

\item Any mesh irregularity can trigger release of physically unstable flow

\item Mesh adaptation only every 12 hours (72 time steps):

\begin{list1}
\item using coarse prediction of future 12 hours
\item gives accurate results
\item reduces remeshing costs
\item reduces likelihood of runaway adaptation
\end{list1}
\end{list0}

\begin{center}\Large\bfseries Future Work \end{center}

\begin{list0}

\item More work on long range prediction of mesh density requirements

\begin{list2}
    \item eg. where will mesh density be required to resolve convection over the
    next few hours
\end{list2}

\item Conservative mapping

\item Gradual anisotropic refinement of polygons

\item Can vertical adaptation be beneficial?

\end{list0}

\end{slide}